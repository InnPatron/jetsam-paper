\documentclass{article}
\usepackage{indentfirst}
\usepackage{ebproof}
\usepackage{amsmath}
\usepackage[
hidelinks,
bookmarksopen
]{hyperref}
\usepackage{listings}

\usepackage{xcolor}
\usepackage{xparse}
\usepackage[style=ACM-Reference-Format,backend=biber]{biblatex}

\addbibresource{references.bib}

\NewDocumentCommand{\codeword}{v}{%
	\texttt{\textcolor{blue}{#1}}%
}

\lstset{
	keywordstyle={\bfseries \color{blue}},
	backgroundcolor={\bfseries \color{gray}},
}


\begin{document}
	
	\tableofcontents
	\pagebreak
	
	\section{Introduction}
	An early-stage specification of a Pyret binding generator for TypeScript code based on augmented simply-typed lambda calculi and source-type indexed logical relations as outlined by \textcite{FFI}.
	
	\section{Core Subset of TypeScript ($TS_{NUM}$)}
	
	A core subset of TypeScript ($TS_{NUM}$) represented as a simply typed lambda calculus augmented with:
	\begin{enumerate}
		\item Double-precision 64-bit binary floats (format IEEE 754)
		\item Float addition (corresponds directly to JS `+' operator)
	\end{enumerate}
	
	\subsection{Terms}
	$t ::=$
	
	\begin{align*}
		float  										&& \text{TS Number} \\
		t_1 + t_2 									&& \text{Addition} \\
		\lambda x\colon T.t_1						&& \text{Abstraction} \\
		t_1 t_2										&& \text{Application}
	\end{align*}


%*********************************************
%*********************************************	
	\subsection{Values}
	v ::=
	\begin{align*}
		float										&& \text{Double-precision 64-bit binary format IEEE 754} \\
		\lambda x\colon T.t							&& \text{Abstraction value}
	\end{align*}
	
	
%*********************************************
%*********************************************	
	\subsection{Types}
	T ::=
	\begin{align*}
		Number												&& \text{Type of floats}\\
		T \rightarrow T										&& \text{Type of functions}
	\end{align*}
	
%*********************************************
%*********************************************
	\subsection{Evaluation}
	
	% E-ADD1
	\begin{prooftree}
		\hypo{t_1 \rightarrow t_1'}
		\infer[right label=@]1[(E-ADD1)]{t_1 + t_2 \rightarrow t_1' + t_2}
	\end{prooftree}\\\\

	%E-ADD2
	\begin{prooftree}
		\hypo{t_2 \rightarrow t_2'}
		\infer[right label=@]1[(E-ADD2)]{v_1 + t_2 \rightarrow v_1 + t_2'}
	\end{prooftree}\\\\

	%E-ADD3
	\begin{prooftree}
		\hypo{float_1 + float_2 = float_3}
		\infer[right label=@]1[(E-ADD3)]{float_1 + float_2 \rightarrow float_3 }
	\end{prooftree}\\\\
	
	%E-APP1
	\begin{prooftree}
		\hypo{t_1 \rightarrow t_1'}
		\infer[right label=@]1[(E-APP1)]{t_1 t_2 \rightarrow t_1' t_2}
	\end{prooftree}\\\\

	%E-APP2
	\begin{prooftree}
		\hypo{t_2 \rightarrow t_2'}
		\infer[right label=@]1[(E-APP2)]{v_1 t_2 \rightarrow v_1 t_2'}
	\end{prooftree}\\\\

	%E-APPABS
	\begin{prooftree}
		\infer[right label=@]0[(E-APPABS)]{
			(\lambda x \colon T.t_1) v_2 \rightarrow [x \mapsto v_2]t_1
		}	
	\end{prooftree}


%*********************************************
%*********************************************
	\subsection{Typing}
	
	% T-NUMBER
	\begin{prooftree}
		\infer[right label=@]0[(T-NUMBER)]{\Gamma \vdash float: Number}
	\end{prooftree}\\\\

	% T-ADD
	\begin{prooftree}
		\hypo{t_1: Number}
		\hypo{t_2: Number}
		\infer[right label=@]2[(T-ADD)]{\Gamma \vdash t_1 + t_2: Number}
	\end{prooftree}\\\\

	% T-VAR
	\begin{prooftree}
		\hypo{x: T \in \Gamma}
		\infer[right label=@]1[(T-VAR)]{\Gamma \vdash x: T}
	\end{prooftree}\\\\

	% T-ABS
	\begin{prooftree}
		\hypo{\Gamma, x: T_1 \vdash t: T_2}
		\infer[right label=@]1[(T-ABS)]{
			\Gamma \vdash \lambda x \colon T_1. t: T_1 \rightarrow T_2
		}
	\end{prooftree}\\\\

	% T-APP
	\begin{prooftree}
		\hypo{\Gamma \vdash t_1: T_1 \rightarrow T_2}
		\hypo{\Gamma \vdash t_2: T_1}
		\infer[right label=@]2[(T-APP)]{
			\Gamma \vdash t_1 t_2: T_2
		}
	\end{prooftree}


%*********************************************
%*********************************************
%*********************************************
%*********************************************
%*********************************************
	\section{Core Subset of Pyret ($PY_{NUM}$)}
	A core subset of Pyret ($PY_{NUM}$) represented as a simply typed lambda calculus augmented with:
	\begin{enumerate}
		\item ``Unlimited'' precision rational numbers (as defined in js-number.js \cite{JSNUMBER})
		\item Rational addition (corresponds to the \lstinline{add()} function found in js-number.js \cite{JSNUMBERSSOURCE})
	\end{enumerate}

	\subsection{Terms}
	$t ::=$
	
	\begin{align*}
	rational  									&& \text{Pyret Number} \\
	t_1 + t_2 									&& \text{Addition} \\
	\lambda x \colon T.t_1								&& \text{Abstraction} \\
	t_1 t_2										&& \text{Application}
	\end{align*}
	
	
	%*********************************************
	%*********************************************	
	\subsection{Values}
	v ::=
	\begin{align*}
		rational							&& \text{``Unlimited'' precision rational} \\
		\lambda x \colon T.t						&& \text{Abstraction value}
	\end{align*}
	
	
	%*********************************************
	%*********************************************	
	\subsection{Types}
	T ::=
	\begin{align*}
		Number												&& \text{Type of Pyret rationals}\\
		T \rightarrow T										&& \text{Type of functions}
	\end{align*}
		
	%*********************************************
	%*********************************************
	Evaluation and typing rules are nearly identical to $TS_{NUM}$ (save for a difference in the numeric type) and are omitted for brevity.


%*********************************************
%*********************************************
%*********************************************
%*********************************************
%*********************************************	
	\section{$L_{v1}$}
	A common target language for $TS_{NUM}$ and $PY_{NUM}$ represented as a simply typed lambda calculus augmented with:
	\begin{enumerate}
		\item ``Unlimited'' precision rational numbers (as found in $PY_{NUM}$)
		\item Rational addition (as found in $PY_{NUM}$)
		\item Double-precision 64-bit binary floats (as found in $TS_{NUM}$)
		\item Float addition (as found in $TS_{NUM}$)
	\end{enumerate}

	\subsection{Terms}
	$t ::=$
	\begin{align*}
		float 		  								&& \text{TS Number} \\
		rational	  								&& \text{Pyret Number} \\
		t_1 + t_2 									&& \text{TS Addition} \\
		t_1 +^{PY} t_2 								&& \text{Pyret Addition} \\
		t_1 t_2 							    	&& \text{Application} \\
		\lambda x\colon T. t					    && \text{Abstraction} \\
	\end{align*}


%*********************************************
%*********************************************
	\subsection{Values}
	v ::=
	\begin{align*}
		float						&& \text{TS Number} \\
		rational					&& \text{Pyret Number}\\
		\lambda x. \tau		    	&& \text{Abstraction} \\
		fail						&& \text{Explicit failure value}
	\end{align*}


%*********************************************
%*********************************************
	\subsection{Types}
	T ::=
	\begin{align*}
		num_f				&& \text{Type of floats}\\
		num_r				&& \text{Type of rationals}\\
		T \rightarrow T		&& \text{Function type}
	\end{align*}
	
	
%*********************************************
%*********************************************	
	\subsection{Evaluation}
	
	Evaluation rules around the `fail' are omitted (evaluation immediately halts).\\
	
	% E-ADD1
	\begin{prooftree}
		\hypo{t_1 \rightarrow t_1'}
		\hypo{t_1' \neq fail}
		\infer[right label=@]2[(E-ADD1)]{t_1 + t_2 \rightarrow t_1' + t_2}
	\end{prooftree}\\\\
	
	% E-ADD2
	\begin{prooftree}
		\hypo{t_2 \rightarrow t_2'}
		\hypo{t_2' \neq fail}
		\infer[right label=@]2[(E-ADD2)]{v_1 + t_2 \rightarrow v_1 + t_2'}
	\end{prooftree}\\\\

	% E-ADD3
	\begin{prooftree}
		\hypo{float_1 + float_2 = float_3}
		\infer[right label=@]1[(E-ADD3)]{float_1 + float_2 \rightarrow float_3}
	\end{prooftree}\\\\

	% E-PYADD1
	\begin{prooftree}
		\hypo{t_1 \rightarrow t_1'}
		\hypo{t_1' \neq fail}
		\infer[right label=@]2[(E-PYADD1)]{t_1 +^{PY} t_2 \rightarrow t_1' +^{PY} t_2}
	\end{prooftree}\\\\
	
	% E-PYADD2
	\begin{prooftree}
		\hypo{t_2 \rightarrow t_2'}
		\hypo{t_2' \neq fail}
		\infer[right label=@]2[(E-PYADD2)]{v_1 +^{PY} t_2 \rightarrow v_1 +^{PY} t_2'}
	\end{prooftree}\\\\
	
	% E-PYADD3
	\begin{prooftree}
		\hypo{float_1 + float_2 = rational_3}
		\infer[right label=@]1[(E-PYADD3)]{float_1 +^{PY} float_2 \rightarrow rational_3}
	\end{prooftree}\\\\

	% E-PYADD4
	\begin{prooftree}
		\hypo{rational_1 + rational_2 = rational_3}
		\infer[right label=@]1[(E-PYADD4)]{rational_1 +^{PY} rational_2 \rightarrow rational_3}
	\end{prooftree}\\\\

	% E-PYADD5
	\begin{prooftree}
		\hypo{rational_1 + float_2 = rational_3}
		\infer[right label=@]1[(E-PYADD5)]{rational_1 +^{PY} float_2 \rightarrow rational_3}
	\end{prooftree}\\\\

	% E-PYADD6
	\begin{prooftree}
		\hypo{float_1 + rational_2 = rational_3}
		\infer[right label=@]1[(E-PYADD6)]{float_1 +^{PY} rational_2 \rightarrow rational_3}
	\end{prooftree}\\\\
	
	The evaluation rules have been omitted for brevity. They are the combination of the evaluations rules of $PY_{NUM}$ and $TS_{NUM}$ (merging application and abstraction). Do note that with such a merging, $float_1 +^{PY} t_2$, etc. would be considered `stuck' with the current formulation.

%*********************************************
%*********************************************
	\subsection{Typing}
	
	% T-FAIL
	\begin{prooftree}
		\infer[right label=@]0[(T-FAIL)]{\Gamma \vdash fail: T}
	\end{prooftree}\\\\
	
	% T-FLOAT
	\begin{prooftree}
		\infer[right label=@]0[(T-FLOAT)]{\Gamma \vdash float: num_f}
	\end{prooftree}\\\\

	% T-RATIONAL
	\begin{prooftree}
		\infer[right label=@]0[(T-RATIONAL)]{\Gamma \vdash rational: num_r}
	\end{prooftree}\\\\
	
	% T-VAR
	\begin{prooftree}
		\hypo{x: T \in \Gamma}
		\infer[right label=@]1[(T-VAR)]{\Gamma \vdash x: T}
	\end{prooftree}\\\\

	% T-ADD
	\begin{prooftree}
		\hypo{\Gamma \vdash t_1: num_f}
		\hypo{\Gamma \vdash t_2: num_f}
		\infer[right label=@]2[(T-ADD)]{\Gamma \vdash t_1 + t_2: num_f}
	\end{prooftree}\\\\

	% T-PY-ADD-1
	\begin{prooftree}
		\hypo{\Gamma \vdash t_1: num_r}
		\hypo{ \Gamma \vdash t_2: num_r}
		\infer[right label=@]2[(T-PY-ADD1)]{\Gamma \vdash t_1 +^{PY} t_2: num_r}
	\end{prooftree}\\\\

	% T-PY-ADD-2
	\begin{prooftree}
		\hypo{\Gamma \vdash t_1: num_r}
		\hypo{\Gamma \vdash t_2: num_f}
		\infer[right label=@]2[(T-PY-ADD2)]{\Gamma \vdash t_1 +^{PY} t_2: num_r}
	\end{prooftree}\\\\

	% T-PY-ADD-3
	\begin{prooftree}
		\hypo{\Gamma \vdash t_1: num_f}
		\hypo{\Gamma \vdash t_2: num_r}
		\infer[right label=@]2[(T-PY-ADD3)]{\Gamma \vdash t_1 +^{PY} t_2: num_r}
	\end{prooftree}\\\\

	% T-ABS
	\begin{prooftree}
		\hypo{\Gamma, x: T_1 \vdash \tau: T_2}
		\infer[right label=@]1[(T-ABS)]{
			\Gamma \vdash \lambda x:T_1. \tau: T_1 \rightarrow T_2
		}
	\end{prooftree}\\\\
	
	% T-APP
	\begin{prooftree}
		\hypo{\Gamma \vdash \tau_1: T_1 \rightarrow T_2}
		\hypo{\Gamma \vdash \tau_2: T_1}
		\infer[right label=@]2[(T-APP)]{
			\Gamma \vdash \tau_1 \tau_2: T_2
		}
	\end{prooftree}

%*********************************************
%*********************************************
%*********************************************
%*********************************************
%*********************************************	
	\section{Compiler $TS_{sub}$ to $L{v1}$}

	\subsection{Types}
	\begin{align*}
		Number					&& = && num_f\\
		T_1 \rightarrow T_2 	&& = && T_1' \rightarrow T_2'
	\end{align*}
	
%*********************************************
%*********************************************	
	\subsection{Terms}
	\begin{align*}
	float: Number							 && = && float: num_f\\
	\tau_1 + \tau_2: Number					 && = && \tau_1' + \tau_2': num_f\\
	\tau_1 \tau_2: T						 && = && \tau_1' \tau_2': T'\\
	\lambda x: T_1.\tau: T_1 \rightarrow T_2 && = && \lambda x: T_1'.\tau': T_1' \rightarrow T_2'
	\end{align*}
	
	
%*********************************************
%*********************************************
%*********************************************
%*********************************************
%*********************************************	
	\section{Compiler $PY_{Number}$ to $L_{v1}$ Target}
	
	\subsection{Types}
	\begin{align*}
	Number					&& = && num_r\\
	T_1 \rightarrow T_2 	&& = && T_1' \rightarrow T_2'
	\end{align*}
	
	%*********************************************
	%*********************************************	
	\subsection{Terms}
	\begin{align*}
	rational: Number						 && = && rational: num_r\\
	\tau_1 + \tau_2: Number					 && = && \tau_1' +^{PY} \tau_2': num_r\\
	\tau_1 \tau_2: T						 && = && \tau_1' \tau_2': T'\\
	\lambda x: T_1.\tau: T_1 \rightarrow T_2 && = && \lambda x: T_1'.\tau': T_1' \rightarrow T_2'
	\end{align*}


%*********************************************
%*********************************************
%*********************************************
%*********************************************
%*********************************************
	\section{$L_{v1}$}
	
	\subparagraph{$TS_{sub}$}
	\begin{align*}
		\mathcal{V}[[Number]]				&& = && \{ n| n \text{ is any float}\}\\
		\mathcal{V}[[\eta_1 \rightarrow \eta_2]]	&& = && \{ 
				(\lambda x.\tau)| 
				\forall v \in \bf{V}[[\eta_1]].[x:=v]\tau \in \mathcal{E}[[\eta_2]]
		\}
	\end{align*}
	
	\subparagraph{$PY_{sub}$}
	\begin{align*}
		\mathcal{V}[[Number]]				&& = && \{ n| n \text{ is any rational}\}\\
		\mathcal{V}[[\theta_1 \rightarrow \theta_2]]	&& = && \{ 
		(\lambda x.\tau)| 
		\forall v \in \bf{V}[[\theta_1]].[x:=v]\tau \in \mathcal{E}[[\theta_2]]
		\}
	\end{align*}
	
	\begin{align*}
		\mathcal{E}[[T]] && = && \{
		\tau | (\tau \Downarrow fail) \lor
		(\tau \Downarrow v \land v \in \mathcal{V}[[T]])
	\}
	\end{align*}
	
	\section{The Convertibility Relationship}
	
	Let $\tau$ be any type from $TS_{NUM}$. Let $\mu$ be any type from $PY_{NUM}$.
	
	\printbibliography[heading=bibintoc]

\end{document}